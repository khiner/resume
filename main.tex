\documentclass[letterpaper,11pt]{article}

\usepackage{latexsym}
\usepackage[empty]{fullpage}
\usepackage{titlesec}
\usepackage{marvosym}
\usepackage[usenames,dvipsnames]{color}
\usepackage{verbatim}
\usepackage{enumitem}
\usepackage{hyperref}
\usepackage{fancyhdr}
\usepackage[english]{babel}
\usepackage{tabularx}
\usepackage{fontawesome}
\usepackage{changepage}

\input{glyphtounicode}

\definecolor{linkcolor}{RGB}{0, 70, 150}

\hypersetup{colorlinks = true}

\pagestyle{fancy}
\fancyhf{} % clear all header and footer fields
\fancyfoot{}
\renewcommand{\headrulewidth}{0pt}
\renewcommand{\footrulewidth}{0pt}

% Adjust margins
\addtolength{\oddsidemargin}{-0.5in}
\addtolength{\evensidemargin}{-0.5in}
\addtolength{\textwidth}{1in}
\addtolength{\topmargin}{-.5in}
\addtolength{\textheight}{1.0in}

\urlstyle{same}

\raggedbottom
\raggedright
\setlength{\tabcolsep}{0in}

% Sections formatting
\titleformat{\section}{
  \vspace{-6pt}\scshape\raggedright\large
}{}{0em}{}[\color{black}\titlerule \vspace{-5pt}]

% Ensure that generate pdf is machine readable/ATS parsable
\pdfgentounicode=1

% Custom commands

\newcommand{\resumeItem}[1]{
  \item\small{
    {#1 \vspace{-2pt}}
  }
}
\newcommand{\resumeSubheading}[4]{
  \item
    \begin{tabular*}{0.97\textwidth}[t]{l@{\extracolsep{\fill}}r}
      \normalsize \textbf{#1} & \small#2 \\
      \small #3 & \small #4 \\
    \end{tabular*}\vspace{-5pt}
}
\newcommand{\resumeEducationHeading}[6]{
  \vspace{-2pt}\item
    \begin{tabular*}{0.97\textwidth}[t]{l@{\extracolsep{\fill}}r}
      \textbf{#1} & \small #2 \\
      \small#3 & \small #4 \\
      \small#5 & \small #6 \\
    \end{tabular*}\vspace{-1.75em}
}
\newcommand{\resumeProjectHeading}[2]{
    \vspace{3pt}\item
    \begin{tabular*}{0.97\textwidth}{l@{\extracolsep{\fill}}r}
      \small#1 & #2 \\
    \end{tabular*}\vspace{-9pt}
}

\renewcommand\labelitemii{$\vcenter{\hbox{\tiny$\bullet$}}$}

\newcommand{\resumeSubHeadingListStart}{\begin{itemize}[leftmargin=0.15in, label={}]}
\newcommand{\resumeSubHeadingListEnd}{\end{itemize}}

\newcommand{\resumeItemListStart}{\begin{itemize}}
\newcommand{\resumeItemListEnd}{\end{itemize}\vspace{-5pt}}

\newcommand{\resumeSectionBody}[1]{
  \vspace{0.7em}
  \small #1
  \vspace{-5pt}
}


\begin{document}

%---------- HEADING ----------

\hypersetup{urlcolor = black} % Don't color link text in the header.

\begin{center}
    \textbf{\Huge Karl Hiner} \\ \vspace{3pt}
    \small
    \faMobile \hspace{.5pt} \href{tel:6127479175}{612-747-9175}
    $|$
    \faAt \hspace{.5pt} \href{mailto:karl.hiner@gmail.com}{karl.hiner@gmail.com}
    $|$
    \faLinkedinSquare \hspace{.5pt} \href{https://www.linkedin.com/in/karl-hiner/}{LinkedIn}
    $|$
    \faGithub \hspace{.5pt} \href{https://github.com/khiner}{GitHub}
    $|$
    \faGlobe \hspace{.5pt} \href{https://karlhiner.com}{Portfolio}
\end{center}
\vspace{-18pt}

\hypersetup{urlcolor = linkcolor} % Change back to link color for the rest of the document.

\section{Education}
  \vspace{1pt}
  \resumeSubHeadingListStart
    \resumeEducationHeading
      {Georgia Institute of Technology}{Atlanta, Georgia}
      {Master's, Computational Science and Engineering $|$ \footnotesize{Expected GPA: 3.9}}{Sep. 2022 \textbf{--} Apr. 2024}{}{}
    \resumeEducationHeading
      {Portland State University}{Portland, Oregon}
      {Bachelor of Science, Computer Science $|$ \footnotesize{GPA: 3.82}}{2010 \textbf{--} 2013}{}{}
  \resumeSubHeadingListEnd

\section{Skills}
  \resumeSubHeadingListStart
    \resumeItem{
        \textbf{Languages:}{ C++, Python, TypeScript/JavaScript, Java, Ruby, Julia, SQL} \\ \vspace{3pt}
        \textbf{Technologies:}{ Git, React, ImGui, Linux, Postgres, PyTorch, JAX, Vulkan/OpenGL, GLM, Node, Docker} \\ \vspace{3pt}
        \textbf{Expertise:}{ Full-stack development, technical leadership, physical modeling and simulation, applied machine learning} \\ \vspace{3pt}
    }
  \resumeSubHeadingListEnd

\section{Experience}
  \resumeSubHeadingListStart
    \resumeSubheading
      {Axiom Data Science}{Portland, Oregon}
      {Lead Software Engineer}{Feb 2020 \textbf{--} Jul 2022}

      \resumeSectionBody{
        Lead developer on the next generation of the Research Workspace, a web application for collaboratively managing data for scientific projects.
        Designed and implemented the following major projects:
      }
      \resumeItemListStart
        \resumeItem{An extensive admin application supporting many internal Research Workspace use cases.}
        \resumeItem{A user authentication and authorization service backend and UI library used across multiple Axiom services.}
        \resumeItem{Internal React packages to modularize and improve UI components, including time pickers and charts.}
        \resumeItem{Data portal features, including a cross-portal feature called \textit{Map Views}, enabling users to create, edit, share, and publish multiple portal map instances.}
        \resumeItem{A remote browser service for internal QA to help quickly discover regressions across portals and landing pages.}
      \resumeItemListEnd
    \resumeSubheading
      {Cozy}{Portland, Oregon}
      {Senior Software Engineer}{Oct 2018 \textbf{--} Jan 2020}
        \resumeItemListStart
            \resumeItem{Led the development of major components of Cozy's web app and payments system.}
            \resumeItem{Transitioned components and services to be presented and consumed by Apartments.com after Cozy's acquisition by CoStar Group.}
            \resumeItem{Mentored junior engineers and led product development efforts on large projects, including project and sprint planning, organizing team efforts, and presenting project progress.}
        \resumeItemListEnd
    \resumeSubheading
      {Self}{Portland, Oregon}
      {Independent Study}{Dec 2017 \textbf{--} Oct 2018}

      \resumeSectionBody{
        I took an unpaid sabbatical to focus on learning more about fields I am passionate about.
        Ultimately, I decided to pursue a Master's degree, aiming to transition into software domains that interest and inspire me.
      }
      \resumeItemListStart
        \resumeItem{Released an \href{https://karlhiner.com/beatbot}{Android app} for sample-based music production.}
        \resumeItem{Studied digital audio signal processing, machine learning, C++, and Python, and produced in-depth Jupyter notebooks for each chapter of eight technical books on these topics.}
        \resumeItem{Completed online courses in deep learning and statistics.}
        \resumeItem{Developed digital audio workstation software.}
        \resumeItem{Implemented a declarative static site generator in React and used it to build my portfolio/blog website.}
      \resumeItemListEnd
    \newpage
    \resumeSubheading
      {New Relic}{Portland, Oregon}
      {Senior Software Engineer - Applied Intelligence Services Team}{Oct 2016 \textbf{--} Dec 2017}
        \resumeItemListStart
          \resumeItem{Researched, architected, and built products, including dynamic baselines, error profiles, and host outlier detection.}
          \resumeItem{Leveraged machine learning and statistical techniques on data from multiple monitoring sources to provide customers with actionable information and context to find, understand, and fix software problems quickly.}
        \resumeItemListEnd
        \vspace{-15pt}
    \resumeSubheading
      {}{}
      {Software Engineer / Senior Software Engineer - Mobile Product Team}{Apr 2014 \textbf{--} Oct 2016}
        \resumeItemListStart
        \resumeItem{Acted as the technical lead on significant features, including activity tracing and version trends, as well as features for crash reporting, network reporting, and real user monitoring.}
        \resumeItem{Designed and shipped APIs, UI features, and high-throughput services using Java, Ruby, and React.}
        \resumeItemListEnd
        \vspace{-15pt}
    \resumeSubheading
      {}{}
      {Junior Software Engineer - Mobile Team}{May 2013 \textbf{--} Apr 2014}
        \resumeItemListStart
          \resumeItem{Co-produced the frontend for the Mobile product Rails application.}
          \resumeItem{Developed data collection and aggregation service features.}
          \resumeItem{Implemented components of the Android application monitoring agent.}
        \resumeItemListEnd
  \resumeSubHeadingListEnd

\section{Relevant Coursework}
  \resumeSubHeadingListStart
    \resumeItem{
      \textbf{GA Tech:}{ Modeling and Simulation, Computational Physics, Computational Data Analysis, Computer Graphics, Computer Animation, Numerical Linear Algebra, High Performance Computing, Machine Learning with Graphs} \\ \vspace{3pt}
      \textbf{PSU:}{ Machine Learning, AI and Game Design, Parallel Programming} \\ \vspace{3pt}
      \textbf{Coursera:}{ Machine Learning, Deep Learning, Probabilistic Graphical Models, Audio Signal Processing} \\ \vspace{3pt}
    }
    \resumeSubHeadingListEnd

\section{Projects}
    \resumeSubHeadingListStart
    \resumeProjectHeading
      {\textbf{MeshEditor} $|$ C++/Vulkan/ImGui $|$ \href{https://github.com/khiner/MeshEditor}{GitHub}}{Nov 2023 \textbf{--} Apr 2024}

      \resumeSectionBody{
        Real-time mesh viewer and editor with rigid body audio modeling, and interactive \href{https://samuelpclarke.com/realimpact/}{RealImpact} dataset explorer supporting comparison of audio models with real-world impact recordings.
      }
    \resumeProjectHeading
      {\textbf{Mesh2Audio} $|$ C++/OpenGL/ImGui $|$ \href{https://github.com/khiner/mesh2audio}{GitHub}}{Jan 2023 \textbf{--} May 2023}

      \resumeSectionBody{
        Real-time modal audio synthesis from 3D meshes, with interactive vertex excitation.
      }
    \resumeProjectHeading
      {\textbf{Drum classification} $|$ Python/PyTorch $|$ \href{https://github.com/khiner/DrumClassification}{GitHub}}{Nov - Dec 2023}

      \resumeSectionBody{
        A drum instrument classification model and preprocessing pipeline for the \href{https://magenta.tensorflow.org/datasets/e-gmd}{Expanded Groove MIDI Dataset} dataset.
      }
    \resumeProjectHeading
      {\textbf{GeoLDMViz} $|$ C++/Python/OpenGL/ImGui $|$ \href{https://github.com/khiner/GeoLDMViz}{GitHub}}{Nov 2023}

      \resumeSectionBody{
        3D visualizer app for inspecting chains of 3D molecules generated with Geometric Latent Diffusion Models.
      }
    \resumeProjectHeading
      {\textbf{Generating Music with WaveNet and SampleRNN} $|$ Python $|$ \href{https://karlhiner.com/music_generation/wavenet_and_samplernn}{GitHub}}{Aug 2019}

      \resumeSectionBody{
        Exploring musical raw audio generation using these popular models.
      }
    \resumeProjectHeading
      {\textbf{FlowGrid} $|$ C++/ImGui $|$ \href{https://github.com/khiner/flowgrid}{GitHub}}{Mar 2022 \textbf{--} Present}

      \resumeSectionBody{
        Immediate-mode interface for the Faust functional audio language, backed by a persistent store supporting navigation to any point in the project history in constant time.
      }
    \resumeProjectHeading
      {\textbf{JAXdsp} $|$ Python/JAX/TypeScript/React $|$ \href{https://github.com/khiner/jaxdsp}{GitHub}}{Dec 2020 \textbf{--} Feb 2022}

      \resumeSectionBody{
        Parameterize audio graphs in real-time to model an incoming/outgoing audio stream pair with differentiable DSP components, with data and audio over WebRTC.
      }
    \resumeProjectHeading
      {\textbf{Jupyter notebooks} $|$ Python $|$ \href{https://github.com/khiner/notebooks}{GitHub}}{Jan 2018 \textbf{--} Jan 2020}

      \resumeSectionBody{
        Python Jupyter notebooks covering each chapter of several books, including:
      }
      \resumeItemListStart
        \resumeItem{Gareth Loy’s Musimathics \href{https://karlhiner.com/jupyter_notebooks/musimathics_volume_1}{Vol 1.} \href{https://karlhiner.com/jupyter_notebooks/musimathics_volume_2}{Vol 2.}}
        \resumeItem{Julius O. Smith’s \href{https://karlhiner.com/jupyter_notebooks/mathematics_of_the_dft}{Mathematics of the DFT}, \href{https://karlhiner.com/jupyter_notebooks/intro_to_digital_filters}{Intro to Digital Filters}, and \href{https://karlhiner.com/jupyter_notebooks/physical_audio_signal_processing}{Physical Audio Signal Processing}}
      \resumeItemListEnd
    \resumeProjectHeading
      {\textbf{BeatBot} $|$ Java/C/OpenGL $|$ \href{https://github.com/khiner/beatbot}{GitHub}}{2012 \textbf{--} 2018}

      \resumeSectionBody{
        A sample-based music production app for Android, with an OpenSL audio/effects backend implemented in C, and a custom OpenGL-based 2D UI designed to minimize draw call submissions for optimal performance on low-end devices.
      }
    \resumeProjectHeading
      {\textbf{Auto-Sampler} $|$ MaxMSP/Ruby/Javascript/C $|$ \href{https://karlhiner.com/music_generation/auto_sampler}{GitHub}}{2015}

      \resumeSectionBody{
        A Max4Live instrument that streams looping audio segments matching the pitch of incoming MIDI notes in real-time.
      }
    \resumeSubHeadingListEnd

% \section{Hobbies}
  % \resumeSectionBody{
    % Watching movies, synthesizers and other music toys, video games, rock climbing, hiking
  % }

%-------------------------------------------
\end{document}
