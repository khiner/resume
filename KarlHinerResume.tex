\documentclass[letterpaper,11pt]{article}

\usepackage{latexsym}
\usepackage[empty]{fullpage}
\usepackage{titlesec}
\usepackage{marvosym}
\usepackage[usenames,dvipsnames]{color}
\usepackage{verbatim}
\usepackage{enumitem}
\usepackage{hyperref}
\usepackage{fancyhdr}
\usepackage[english]{babel}
\usepackage{tabularx}
\usepackage{fontawesome}
\usepackage{changepage}

\input{glyphtounicode}

\definecolor{linkcolor}{RGB}{0, 70, 150}

\hypersetup{colorlinks = true}

\pagestyle{fancy}
\fancyhf{} % clear all header and footer fields
\fancyfoot{}
\renewcommand{\headrulewidth}{0pt}
\renewcommand{\footrulewidth}{0pt}

% Adjust margins
\addtolength{\oddsidemargin}{-0.5in}
\addtolength{\evensidemargin}{-0.5in}
\addtolength{\textwidth}{1in}
\addtolength{\topmargin}{-.5in}
\addtolength{\textheight}{1.0in}

\urlstyle{same}

\raggedbottom
\raggedright
\setlength{\tabcolsep}{0in}

% Sections formatting
\titleformat{\section}{
  \vspace{-6pt}\scshape\raggedright\large
}{}{0em}{}[\color{black}\titlerule \vspace{-5pt}]

% Ensure that generate pdf is machine readable/ATS parsable
\pdfgentounicode=1

% Custom commands

\newcommand{\resumeItem}[1]{
  \item\small{
    {#1 \vspace{-2pt}}
  }
}
\newcommand{\resumeSubheading}[4]{
  \item
    \begin{tabular*}{0.97\textwidth}[t]{l@{\extracolsep{\fill}}r}
      \normalsize \textbf{#1} & \small#2 \\
      \small #3 & \small #4 \\
    \end{tabular*}\vspace{-9pt}
}
\newcommand{\resumeEducationHeading}[6]{
  \vspace{-2pt}\item
    \begin{tabular*}{0.97\textwidth}[t]{l@{\extracolsep{\fill}}r}
      \textbf{#1} & \small #2 \\
      \small#3 & \small #4 \\
      \small#5 & \small #6 \\
    \end{tabular*}\vspace{-1.75em}
}
\newcommand{\resumeProjectHeading}[2]{
    \vspace{3pt}\item
    \begin{tabular*}{0.97\textwidth}{l@{\extracolsep{\fill}}r}
      \small#1 & #2 \\
    \end{tabular*}\vspace{-9pt}
}

\renewcommand\labelitemii{$\vcenter{\hbox{\tiny$\bullet$}}$}

\newcommand{\resumeSubHeadingListStart}{\begin{itemize}[leftmargin=0.15in, label={}]}
\newcommand{\resumeSubHeadingListEnd}{\end{itemize}}

\newcommand{\resumeItemListStart}{\begin{itemize}}
\newcommand{\resumeItemListEnd}{\end{itemize}\vspace{-5pt}}

\newcommand{\resumeSectionBody}[1]{
  \vspace{0.7em}
  \small #1
  \vspace{-5pt}
}


\begin{document}

%---------- HEADING ----------

\hypersetup{urlcolor = black} % Don't color link text in the header.

\begin{center}
    \textbf{\Huge Karl Hiner} \\ \vspace{3pt}
    \small
    \faMobile \hspace{.5pt} \href{tel:6127479175}{612-747-9175}
    $|$
    \faAt \hspace{.5pt} \href{mailto:karl.hiner@gmail.com}{karl.hiner@gmail.com}
    $|$
    \faLinkedinSquare \hspace{.5pt} \href{https://www.linkedin.com/in/karl-hiner/}{LinkedIn}
    $|$
    \faGithub \hspace{.5pt} \href{https://github.com/khiner}{GitHub}
    $|$
    \faGlobe \hspace{.5pt} \href{https://karlhiner.com}{Portfolio}
\end{center}
\vspace{-18pt}

\hypersetup{urlcolor = linkcolor} % Change back to link color for the rest of the document.

\section{Education}
  \vspace{1pt}
  \resumeSubHeadingListStart
    \resumeEducationHeading
      {Georgia Institute of Technology}{Atlanta, Georgia}
      {Master of Science, Computational Science and Engineering $|$ \footnotesize{GPA: 3.90}}{Sep 2022 \textbf{--} Apr 2024}{}{}
    \resumeEducationHeading
      {Portland State University}{Portland, Oregon}
      {Bachelor of Science, Computer Science $|$ \footnotesize{GPA: 3.82}}{2010 \textbf{--} 2013}{}{}
  \resumeSubHeadingListEnd

\section{Experience}
  \resumeSubHeadingListStart
    \resumeSubheading
      {Axiom Data Science}{Portland, Oregon}
      {Lead Software Engineer}{Feb 2020 \textbf{--} Jul 2022}

      \resumeSectionBody{
        Lead developer on the next generation of the Research Workspace, a web application for collaboratively managing data for scientific projects.
      }
      \resumeItemListStart
        \resumeItem{Developed an extensive admin application and near-complete rewrite of the Research Workspace, dramatically simplifying application architecture while expanding capabilities, using TypeScript, GraphQL, and Postgres.}
        \resumeItem{Implemented a user authentication and authorization service and UI library to replace several existing services across applications, integrating with Google auth and supporting flexible roles and feature access controls. Amazon Cognito, Express.js, and Postgres for the backend, and TypeScript/React for the client package and UI.}
        \resumeItem{Developed extensible internal React packages to modularize and modernize UI components, including charts and time pickers, including Storybook UI component tests and documentation, and replaced legacy data portal components and systems in a piecemeal migration from Backbone to React.}
        \resumeItem{Implemented full-stack data portal features, including a cross-portal feature called \textit{Map Views}, enabling users to create, edit, share, and publish multiple portal map instances.}
        \resumeItem{Built a remote browser service and UI for internal QA to quickly discover regressions across data portals.}
      \resumeItemListEnd
    \resumeSubheading
      {Cozy}{Portland, Oregon}
      {Senior Software Engineer}{Oct 2018 \textbf{--} Jan 2020}

        \resumeItemListStart
          \resumeItem{Led the transition of major components and services to be presented and consumed by Apartments.com after Cozy's acquisition by CoStar Group using Ruby and Javascript.}
          \resumeItem{Implemented a Stripe-backed payment method creation flow, including detailed error handling and user-facing notifications, improving the user experience and reducing support volume.}
          \resumeItem{Integrated Stripe Radar risk evaluations into fraud detection pipeline, improving fraud detection accuracy.}
          \resumeItem{Automated a manual process for determining landlord tax requirements by aggregating data from multiple sources, reducing the time to determine tax requirements from hours to seconds.}
          \resumeItem{Implemented a model to predict and flag accounts that are likely to fail their next payment, preventing ACH payment failures and reducing support volume.}
          \resumeItem{Mentored junior engineers and led development efforts on major projects, led sprint planning, planned and organized team efforts, and presented team progress.}
        \resumeItemListEnd
    \resumeSubheading
      {Self}{Portland, Oregon}
      {Independent Study}{Dec 2017 \textbf{--} Oct 2018}

      \resumeSectionBody{
        Dedicated time to advancing my expertise in machine learning, audio signal processing, and software development.
        This was the beginning of a professional development journey that eventually led me to pursue my Master's degree at Georgia Tech.
      }
      \resumeItemListStart
        \resumeItem{Released \href{https://karlhiner.com/beatbot}{BeatBot}, a sample-based music production app for Android with a C-based audio/effects backend and a custom OpenGL 2D UI designed to minimize draw call submissions for optimal performance on low-end devices.}
        \resumeItem{Studied digital audio signal processing, machine learning, C++, and Python, and produced in-depth Jupyter notebooks for each chapter of eight technical books on these topics.}
        \resumeItem{Using C++ and the JUCE framework, developed a node-based \href{https://github.com/khiner/flowgrid_old}{digital audio workstation} that automatically determines default connections based on a grid layout, supporting VST plugins and audio/MIDI input/output, integrating with the Push 2 MIDI controller and LCD display.}
        \resumeItem{Completed deep learning and statistics online courses.}
        \resumeItem{Implemented a declarative static site generator in React and used it to build my portfolio website.}
      \resumeItemListEnd
    \newpage
    \resumeSubheading
      {New Relic}{Portland, Oregon}
      {Senior Software Engineer - Applied Intelligence Services Team}{Oct 2016 \textbf{--} Dec 2017}

        \resumeItemListStart
          \resumeItem{Researched, architected, built, and shipped products using Java, React, and Ruby, including dynamic baselines, error profiles, and host outlier detection.}
          \resumeItem{Leveraged statistical techniques on data from multiple monitoring sources to extract actionable insights, enabling users to understand and fix complex software problems quickly.}
          \resumeItem{Prototyped novel time-series algorithms and data visualizations using Python, Jupyter, D3.js, and Three.js, including a T-SNE-based particle visualization for host clustering and outlier detection that was demoed at the main New Relic conference, FutureStack.}
        \resumeItemListEnd
        \vspace{-15pt}
    \resumeSubheading
      {}{}
      {Software Engineer / Senior Software Engineer - Mobile Product Team}{Apr 2014 \textbf{--} Oct 2016}
        \resumeItemListStart
        \resumeItem{Acted as the technical lead on significant features, including activity tracing and version trends.}
        \resumeItem{Migrated the crash reporting database and API from Postgres to Cassandra, dramatically reducing write and query times while increasing data retention.}
        \resumeItem{Implemented major features for crash reporting, network reporting, and real user monitoring using Java, React, and Ruby.}
        \resumeItemListEnd
        \vspace{-15pt}
    \resumeSubheading
      {}{}
      {Junior Software Engineer - Mobile Team}{May 2013 \textbf{--} Apr 2014}

        \resumeItemListStart
          \resumeItem{Co-produced the frontend for the Mobile product Ruby on Rails application.}
          \resumeItem{Developed features for the data collection and aggregation service backend and the Android application monitoring agent, receiving billions of data harvest posts per day.}
        \resumeItemListEnd
  \resumeSubHeadingListEnd
\vspace{-10pt}
\section{Projects}
    \resumeSubHeadingListStart
    \resumeProjectHeading
      {\href{https://github.com/khiner/MeshEditor}{\textbf{MeshEditor}} $|$ C++/Vulkan/ImGui}{Nov 2023 \textbf{--} Apr 2024}

      \resumeSectionBody{
        Real-time mesh viewer and editor with rigid body audio modeling, and interactive \href{https://samuelpclarke.com/realimpact/}{RealImpact} dataset explorer supporting comparison of audio models with real-world impact recordings.
      }
    \resumeProjectHeading
      {\href{https://github.com/khiner/mesh2audio}{\textbf{Mesh2Audio}} $|$ C++/OpenGL/ImGui}{Jan 2023 \textbf{--} May 2023}

      \resumeSectionBody{
        Real-time modal audio synthesis from 3D meshes, with interactive vertex excitation.
      }
    \resumeProjectHeading
      {\href{https://github.com/khiner/ProcessingRayTracer}{\textbf{Processing Ray Tracer}} $|$ Java/Processing}{2024}

      \resumeSectionBody{
        Ray tracing, implicit surface generation, and mesh manipulation projects, written in the Processing (Java) language.
      }
    \resumeProjectHeading
      {\href{https://github.com/khiner/DrumClassification}{\textbf{Drum classification}} $|$ Python/PyTorch}{Dec 2023}

      \resumeSectionBody{
        A drum instrument classification model and preprocessing pipeline for the \href{https://magenta.tensorflow.org/datasets/e-gmd}{Expanded Groove MIDI Dataset} dataset.
      }
    \resumeProjectHeading
      {\href{https://github.com/khiner/GeoLDMViz}{\textbf{GeoLDMViz}} $|$ C++/Python/OpenGL/ImGui}{Nov 2023}

      \resumeSectionBody{
        3D visualizer app for inspecting chains of 3D molecules generated with Geometric Latent Diffusion Models.
      }
    \resumeProjectHeading
      {\href{https://github.com/khiner/flowgrid}{\textbf{FlowGrid}} $|$ C++/ImGui}{Mar 2022 \textbf{--} Present}

      \resumeSectionBody{
        Immediate-mode editor for Faust (functional audio language) programs, backed by a persistent store supporting constant-time navigation to any point in project history.
        Includes a from-scratch syntax aware embedded text editor with a language-complete \href{https://github.com/khiner/tree-sitter-faust}{Faust tree-sitter grammar}, editing LLVM JIT-compiled Faust,
        a highly configurable/monitorable audio graph editor and matrix mixer,
        complete implementation of Faust DSP UI spec, and extensive audio device config.}
    \resumeProjectHeading
      {\href{https://github.com/khiner/notebooks}{\textbf{Jupyter notebooks}} $|$ Python}{Jan 2018 \textbf{--} Jan 2020}

      \resumeSectionBody{
        Python Jupyter notebooks covering each chapter of several books, including Gareth Loy’s
        Musimathics \href{https://karlhiner.com/jupyter_notebooks/musimathics_volume_1}{Vol 1.}/\href{https://karlhiner.com/jupyter_notebooks/musimathics_volume_2}{Vol 2.}
        and Julius O. Smith’s \href{https://karlhiner.com/jupyter_notebooks/mathematics_of_the_dft}{Mathematics of the DFT}, \href{https://karlhiner.com/jupyter_notebooks/intro_to_digital_filters}{Intro to Digital Filters}, and \href{https://karlhiner.com/jupyter_notebooks/physical_audio_signal_processing}{Physical Audio Signal Processing}.
      }
    \resumeProjectHeading
      {\href{https://github.com/khiner/jaxdsp}{\textbf{JAXdsp}} $|$ Python/JAX/TypeScript/React}{Dec 2020 \textbf{--} Feb 2022}

      \resumeSectionBody{
        Parameterize audio graphs in real-time to model an incoming/outgoing audio stream pair with differentiable DSP components, with data and audio over WebRTC.
      }
    \resumeProjectHeading
      {\href{https://karlhiner.com/music_generation/wavenet_and_samplernn}{\textbf{Generating Music with WaveNet and SampleRNN}} $|$ Python}{Aug 2019}

      \resumeSectionBody{
        Exploring musical raw audio generation using these popular models.
      }
    \resumeProjectHeading
      {\href{https://github.com/khiner/beatbot}{\textbf{BeatBot}} $|$ Java/C/OpenGL}{2012 \textbf{--} 2018}

      \resumeSectionBody{
        A sample-based music production app for Android, with an OpenSL audio/effects backend implemented in C, and a custom OpenGL-based 2D UI designed to minimize draw call submissions for optimal performance on low-end devices.
      }
    % \resumeProjectHeading
    %   {\href{https://karlhiner.com/music_generation/auto_sampler}{\textbf{Auto-Sampler}} $|$ MaxMSP/Ruby/Javascript/C}{2015}

    %   \resumeSectionBody{
    %     A Max4Live instrument that streams looping audio segments matching the pitch of incoming MIDI notes in real-time.
    %   }
    \resumeSubHeadingListEnd

\section{Relevant Coursework}
  \resumeSubHeadingListStart
    \resumeItem{
      \textbf{GA Tech:}{ Modeling and Simulation, Computational Physics, Computational Data Analysis, Computer Graphics, Computer Animation, Numerical Linear Algebra, High Performance Computing, Machine Learning with Graphs} \\ \vspace{3pt}
      \textbf{PSU:}{ Machine Learning, AI and Game Design, Parallel Programming} \\ \vspace{3pt}
      \textbf{Coursera:}{ Machine Learning, Deep Learning, Probabilistic Graphical Models, Audio Signal Processing} \\ \vspace{3pt}
    }
    \resumeSubHeadingListEnd

\end{document}
